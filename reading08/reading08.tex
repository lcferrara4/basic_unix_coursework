\documentclass[letterpaper]{article}

\title{Reading 08: Document Tools}
\date{}
\author{Lauren Ferrara}

\usepackage{graphicx}
\usepackage{hyperref}
\usepackage[margin=1in]{geometry}

\begin{document}

\maketitle

%------------------------------------------------------------------------------

\section*{Overview}

For this experiment, I created {\bf three} scripts:

\begin{enumerate}

\item{\tt roll\_dice.sh}: This script simulates rolling a dice.

\item{\tt experiment.sh}: This script uses roll {\tt dice.sh} to perform an experiment and then collect that data into {\tt results.dat}.

\item{\tt histogram.plt}: This script uses gnuplot to create a graph of the data in {\tt results.dat}.

\end{enumerate}

%------------------------------------------------------------------------------

\section*{Rolling Dice}

First, I created a script called roll {\tt dice.sh} that uses the shuf command to simulate rolling a die with a certain number of {\it sides} for a specified amount of {\it rolls}.

\begin{verbatim}
$ ./roll_dice.sh -h
usage: roll_dice.sh [-r ROLLS -s sides]

-r ROLLS Number of rolls of die (default: 10)
-s SIDES Number of sides on die (default: 6)
\end{verbatim}

%------------------------------------------------------------------------------

\section*{Experiment}

Second, I created a script called {\tt experiment.sh} that uses {\tt roll\_dice.sh} to simulate rolling a {\bf six}-sided die 1000 times. My script uses {\tt awk} to collect the results into a single file called {\tt results.txt}.

%------------------------------------------------------------------------------

\section*{Results}

Table 1 contains the results of my experiment of rolling a dice 1000 times:

\begin{table}[h!]
    \centering
    \begin{tabular}{r||c}
    Side	& Counts\\
    \hline
    1	& 161\\
    2	& 167\\
    3	& 156\\
    4	& 177\\
    5	& 163\\
    6	& 176\\
    \end{tabular}
    \caption{Dice Rolling Results}
    \label{tbl:dice}
\end{table}

Figure 1 contains a plot of my experimental results as produced by {\tt histogram.plt}:

\begin{figure}[h!]
  \centering
  \includegraphics[width=5in]{results.png}
  \caption{Dice Rolling Results}
  \label{fig:results}
\end{figure}

\end{document}
